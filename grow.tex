\section{Rosenbluth algorithm}
Polymers are grown using the Rosenbluth algorithm \cite{rosenbluth1955monte}. Between the beads we take a fixed distance of 1. The first two bead of a polymer are placed at positions $(0,0)$ and $(1,0)$. For the third bead, a discrete number of candidate positions $\theta_j$ are determined. In our case $j=1,\cdots,6$. Since the distance between neighbouring beads is always one, these positions lie on a circle around the first bead, with a spacing of $2\pi/6$ between them and with a random offset. 
\begin{gather}
    w_j^{(l)} = e^\frac{-E(\theta_j)}{k_BT},
\end{gather} where $l$ denotes the bead that will be added, $E(\theta_j)$ is the energy associated with candidate position $\theta_j$. The sum of the weights $w_j^{(l)}$ from the candidate positions is $W^{(l)} = \sum_j w_j^{(l)}$. The final position of bead $l$ is then chosen with probability $w_j^{(l)}/W^{(l)}$. This is done using the earlier mentioned roulette-wheel algorithm. 

Not all polymer configurations are equally likely, to this end the polymers have weights assigned to them. When a polymer of $L$ beads is fully grown, the weight of the polymer is $\prod_{l=2}^L w_j^{(L)}$ ($l=2$ is the third polymer, since the simulation starts counting at $0$). Trivially a less likely polymer configuration will have lower weight then a more likely configuration.

The energy between the

\begin{gather}
    V_{LJ} = 4\epsilon \left[ \left(\frac{\sigma}{r}\right)^{12} - \left(\frac{\sigma}{r}\right)^{6} \right] ,
\end{gather} where $\epsilon$ is the depth of the potential well, $\sigma$ is the finite distance where the potential is zero and $r$ is the distance between two particles. The parameters chosen for the LJ potential are $\epsilon=0.25$ and $\sigma=0.8$.