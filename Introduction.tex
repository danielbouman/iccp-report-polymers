\section{Introduction}


\subsection{Markov chain}
For a canonical ensemble the probability of a given state $X$ is proportional to $P(X) \propto e^{-\beta E(X)}$. However, for states which have high energies $E$, the probabilities $P$ become vanishingly small. For computional reasons then, a bias for lower energies should be implemented when generating configurations. For polymers, it is natural to generate configurations on a per monomer basis. The energy difference in adding a single monomer, and so also the Boltzmann weight, is dependent only on previously placed monomers. This means we can model the system as a Markov chain.

\subsection{Rosenbluth and Rosenbluth algorithm}
In $n$ dimensions, if the distance between each monomer is set to 1, the $N_{\theta}$ amount of possible new monomer positions are positioned on a ($n$-1)-sphere of unit radius, with the last monomer as the centre.