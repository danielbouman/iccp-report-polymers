\section{Introduction}

<<<<<<< HEAD

By assuming the polymers are situated in a dilute solution, polymer-polymer self-interactions will prevail. This means that to extract data of the behaviour of the polymers, creating an ensemble of individual polymers will be sufficient.
The interaction between the beads is modelled by a Lennard-Jones (LJ) potential (Eq. \ref{eq:lj}, which represents a Van der Waals attraction at larger distances and repulsion at smaller distances, and a bending energy (Eq. \ref{eq:bending_energy}), which simulates the semi-flexibility of polymers. The LJ potential energy is given by 

\begin{equation}\label{eq:lj}
   E_{LJ} = 4\epsilon \left[ \left(\frac{\sigma}{r}\right)^{12} - \left(\frac{\sigma}{r}\right)^{6} \right],
\end{equation} where $\epsilon$ is the depth of the potential well, $\sigma$ is the finite distance where the potential is zero and $r$ is the distance between two particles. The parameters chosen for the LJ potential are $\epsilon=0.25$ and $\sigma=0.8$ \cite{jmt}. The bending energy is given by
\begin{equation}\label{eq:bending_energy}
    E_{b} = \epsilon_b(1-\cos{\theta}),
\end{equation} where $\epsilon_b\geq 0$ is the bending energy strength and $\theta$ is the angle between the new and preceding bead \cite{hsu2011review}.

\subsection{Markov chain}
For a canonical ensemble the probability of a given state $\alpha$ is proportional to the Boltzmann weight, $P(\alpha) \propto e^{-\beta E(\alpha)}$, where the proportionality constant is given by the normalization constraint.  Here $\beta=1/\left(k_B T\right)$, where $k_B$ is the Boltzmann constant, $T$ the temperature and $E(\alpha)$ the energy of a state $\alpha$. For polymers, it is natural to generate configurations on a per bead basis. The only restriction on the each new 

The energy gained by adding a monomer is dependent only on previously placed monomers. This means the simulation acts like a Markov chain. The basic Rosenbluth and Rosenbluth (RR) algorithm outlined below follows a Markov Chain Monte Carlo method.




%\subsection{Roulette-wheel algorithm}
%The roulette-wheel algorithm, also known as fitness proportionate selection, is a method to select a member $j$ from a population with $N$ members, with a certain probability $p_j$. This is done by dividing the interval $[0,1]$ into $N$ segments of size $p_j$. Then a uniform random number is generated in that same interval. A member is selected by checking in which interval the random number lies.


%\subsection{Rosenbluth and Rosenbluth algorithm}
%In $d$ dimensions, if the distance between each monomer is set to 1, the amount of possible new monomer positions $N_{\theta}$ are positioned on a ($d$-1)-sphere of unit radius, with the last monomer as the centre.
