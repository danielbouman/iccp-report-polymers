\section{Pruning and enrichment}
For long chains the Rosenbluth method becomes inefficient, as was shown by Batoulis and Kremer \cite{batoulis1988statistical}. To make the Rosenbluth algorithm more efficient, Grassberger has suggested pruning and enriching the population. If a partially grown polymer gets a weight below a certain threshold, half of the time the polymer is discarded (`pruned'), and the other half of the time its weight is doubled (to compensate for the pruned polymers) and growing continues. When a partially grown polymer gets a weight above a certain threshold, the polymer is split. Both polymers are independently grown from this point on and their weights are halved. The population is now `enriched'. This combined with the Rosenbluth algorithm is called the pruned-enriched Rosenbluth method (PERM).

The reasoning behind pruning and enrichment is that it is inefficient to continue growing a partial polymer with a low weight since it will have a low contribution to the weighted average. It is then better to prune this polymer and start over. When a polymer gets a relatively high weight it is important and advantageous to make a copy of this polymer.

\todo[inline]{Illustration of pruning and enrichment}

For a finite system the partition sum can be estimated by 
\begin{gather*}
    Z^{(l)} \simeq \hat{Z}^{(l)}
    = \frac{1}{N_l} \sum_{n=0}^{N_l}W_n^{(l)},
\end{gather*} where $N_l$ is the number of thus far completed polymers. The partition function is used to determine the upper limit $W_n^+$ and lower limit $W_n^-$ for the polymer weights, above or below which a polymer is enriched or pruned respectively as described by Hsu and Grassberger\cite{hsu2011review}. This is done as follows
\begin{gather*}
    W_+^{(l)} = C_+\hat{Z}^{(l)},\\
    W_-^{(l)} = C_-\hat{Z}^{(l)},
\end{gather*} where $C_+$ and $C_-$ are constants of order unity and should have a ratio $C_+/C_- \simeq \mathcal{O}(10)$. These constants determine if the population will grow or shrink and are chosen by running the simulation with different values and choosing the best combination.

For the first polymer there are no $\hat{Z}^{(l)}$ yet, so for this polymer the upper and lower limits are set to infinity and zero respectively ($ W_+^{(l)}= \infty$ and $W_-^{(l)}=0$). Resulting in the regular Rosenbluth algorithm.

The expectation value of a physical quantity $A$ can be determined by 
\begin{equation}
    \langle A \rangle^{(l)} = \frac{\sum_{n=0}^{N_l}A_nW_n^{(l)}}{\sum_{n=0}^{N_l}W_n^{(l)}}.
\end{equation}
