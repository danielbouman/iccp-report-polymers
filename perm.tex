\section{Pruning and enrichment}
For long chains the Rosenbluth method becomes inefficient, as was shown by Batoulis and Kremer \cite{batoulis1988statistical}. To make the Rosenbluth algorithm more efficient, Grassberger has suggested pruning and enriching the population. If a partially grown polymer gets a weight below a certain threshold, it is discarded or 'pruned'. When a partially grown polymer gets a weight above a certain, the polymer is split. Both polymers are independently grown from this point on. The population is now 'enriched'. This combined with the Rosenbluth algorithm is called pruned-enriched Rosenbluth method (PERM). 

The reasoning behind pruning and enrichment is that it is inefficient to continue growing a partial polymer with a low weight since it will have a low contribution to the weighted average. It is then better to prune this polymer and start over. When a polymer gets a relatively high weight it is important and advantageous to make a copy of this polymer.

The partition sum of a canonical ensemble in thermal equilibrium is given by
\begin{gather*}
    Z(\beta) = \sum_X e^{-\beta E(X)}.
\end{gather*}
For a finite system the partition sum can be estimated by \cite{hsu2011review}
\begin{gather*}
    Z_n \simeq \hat{Z}_n
    = \frac{1}{N_l} \sum_{n=0}^{N_l}W_l(n),
\end{gather*} where $N_l$ is the number of completed polymers, 

\begin{gather*}
    W_n^+ = C_+\hat{Z}_n,\\
    W_n^- = C_-\hat{Z}_n,
\end{gather*} where $C_+$ and $$

$C_+/C_- \simeq \O(10)$

For the first polymer there are no 
